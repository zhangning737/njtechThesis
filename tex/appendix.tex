\chapter{南京理工大学博士、硕士学位论文撰写格式}
\label{app:format}
为了规范博士、硕士学位论文的撰写,根据由国家标准局批准颁发的GB7713-87《科学技术报告、学位论文和学术论文的编写格式》,将博士、硕士学位论文的编写格式及有关标准统一规定如下:

详细参见研究生网站(http://gs.njust.edu.cn/a/xwgl/xwsq/20130508/82.html)。

\section{学位论文的装订}

博士、硕士学位论文页面设置一律为:上空30mm,下空24mm,左空25mm,右空25mm,对称页边距,页眉20mm,页脚20mm。用A4(297mm$\times $210mm)标准大小的白纸,装订成册后尺寸为(292mm$\times$207mm)。硕士论文封面用157g白色铜版纸,博士论文封面用230g黄色云彩纸。
封二、英文封二、声明和学位论文使用声明采用单页印刷,从中文摘要开始采用双面印刷。
正文中的一级标题(章目)用小3号加粗宋体,段前段后各空18磅,居左;二级标题(条)用4号加粗宋体,段前段后各空12磅,居左;三级标题(款)用小4号加粗宋体,段前段后各空6磅,居左;四级标题(项)同正文用小4号宋体,行距20磅。数字和字母采用Times New Roman体。样本详见附件。

学位论文格式为:( 注:页眉字体为小5号宋体)

奇数页眉:
博士/硕士论文\hspace{40pt}论文题目

偶数页眉:
章节号和名\hspace{40pt}/硕士论文


\section{学位论文前置部分}
\subsection{封面}
封面按统一的博士、硕士学位论文封面的内容和格式填写。(见附件一,注:密级部分如:秘密、机密或绝密必须填,其余可不填。密级后面★作标志,★后注明保密期限。)
书脊要注明学位论文题名及学位授予单位名称。

\subsection{封二}
学位论文的封二可作为封面标识项目的延续,内容包括学位论文级别、题目、作者、指导教师、作者单位、出版时间等。该页置于封面下面,包括中英文版,中文在前,英文在后。字体和字号以封面为准。见附件三和附件四。

\subsection{声明}
另页起,用附件五,对其内容不得作任何改动。该声明置于封二之后,中文摘要之前。

\subsection{摘要}
摘要是学位论文内容的不加注释和评论的简短陈述,说明研究工作的目的、实验方法、实验结果和最终结论等。应是一篇完整的短文,可以独立使用和引用,摘要中一般不用图表、化学结构式和非公知公用的符号和术语。标题用3号宋体加粗,居中,正文用小4号宋体。
摘要分中、外两篇,硕士论文摘要中文字数400~600个字,博士论文摘要中文字数800~1000个字。英文摘要的内容与中文摘要一致,且需合符语法,语句通顺。
摘要的装订按中、外文顺序进行,置于声明之后,分别由另页开始。详见附件六、七。

\subsection{关键词}
关键词是为了便于文献标引从该学位论文中选取出来用以表示全文主题内容信息款目的单词或术语,一般选取3~8个。其中关键词三个字用四号宋体加粗,其余用小4号宋体。
关键词写法的例:关键词:专家系统,模糊数学,枪械设计,知识库
关键词分为中、外文分别附在中、外文摘要的末尾。见附件六、七。

\subsection{目次页}
目次页由学位论文的一、二、三级标题、致谢、参考文献、附录等的序号、名称和页码组成,目次页置于外文摘要后,由另页开始。其中目录两字用3号宋体加粗,一级标题、致谢、参考文献、附录等用4号加粗宋体,其余为小4号宋体。见附件八。

\subsection{图表清单}
如遇图表较多,可以分别列出清单,清单置于目次页后,由另页开始。本条为非必要部分。图的清单应有序号、图名和页码。表的清单应有序号、表名和页码。图表清单置于目次页之后,由另页开始。“图表目录”四字用3号宋体加粗,其余用小4号宋体。正文中表说用5号宋体在表上,图说用5号宋体在图下,图表内的字体用5号宋体。如论文中无图表,此项可免。见附件九。

\subsection{注释表}
注释表为符号、标志、缩略词、首字母缩写、计量单位、名词和术语等的注释说明汇集表,置于图表清单后,由另页开始,本条为非必要部分。

\subsection{注释}
当论文中的字、词或短语需要进一步加以说明或标明具体的文献来源时,用注释。注释采取集中著录在“文后”或分散著录在“脚注”。

论文“文后”集中注释示例:
①国家自然科学基金项目(30070218)。
②傅深渊(1963-),男,浙江省XX人,毕业于XX大学XX专业,……。 

…………。
论文“脚注”分散著录注释示例:
①国家自然科学基金项目(30070218)。
②傅深渊(1963-),男,浙江省XX人,毕业于XX大学XX专业,……。

…………。

\section{学位论文的主要部分}
详见附件十、十一、十二、十三。

\subsection{引言(或绪论)}
引言(或绪论)简要说明研究工作的目的、范围、前人的工作和知识空白、理论基础和分析、研究设想、研究方法、实验设计、预期结果和意义等。引言(或绪论)不要与摘要雷同,一般教科书有的知识,在引言中不必赘述。
引言(或绪论)重点应放在有关历史回顾和前人工作的综合评述,以及理论分析等,应该用足够的文字叙述,如有必要可单独编成第一章。详见附件十。

\subsection{正文}
正文是学位论文的核心部分,要求做到客观真切,准确完备,合乎逻辑,层次分明,简练可读。
正文按一级标题(章)、二级标题(条)、三级标题(款)、四级标题(项)的次序编排,其中的图表等的序号归入本身所处的本层次的次序中(见图3.2.1)

\subsubsection{图}
图应有图名、图号及必要的说明。
图应具有“自明性”,即只看图、图名和图例,不阅读正文,就可理解图意。必要时应将图上的符号、标记、代码或实验条件等用最简练的文字,横排于图名的下方。曲线图的纵横坐标应标注量纲及标准规定的符号,只有在不必要标明(如无量纲等)的情况下方可省略。
图号按章编排,图名在图号之后空一格排写,图中若有分图时用a)、b)等置于分图之下。如第四章第一个图的图号及图名:图4.1 和通关系证明示意图

。。。

