\chapter{结论与展望}
\section{结论}

本文通过对盐岩国内外研究的调研,总结盐岩力学特性以及盐岩蠕变常用本构模型,了解盐岩利用、研究的历程、目前所面临的难题,当下主要的研究方向等。选用盐岩RGBa蠕变模型,对盐岩热力耦合方程进行推导演算。依托江苏金坛盐岩储库工程,建立计算模型,并制定不同类型储库的不同工况进行计算模拟,主要结论如下:

(1)通过研究现状的调研发现,目前盐岩储气库的研究主要方向是盐岩储气库的综合评价准则,以及在保证安全稳定前提下,盐岩储库经济效益最大化的实现方法。通过数值模拟,研究储库长期运行过程中热力场与应力场之间的相互作用,以此对储库的选址、设计细节、运维管理等相关研究提供参考依据。

(2)相同边界条件下,考虑热力耦合的深部地下盐岩新能源储库,运行\num{10}年后,围岩最大位移为\SI{0.336}{m};不考虑热力耦合的围岩最大位移为\SI{0.315}{m},相差约为6.6%。相同边界条件下,考虑热力耦合的深部地下盐岩传统能源储库,运行\num{10}年后,围岩最大位移为\SI{0.375}{m},不考虑热力耦合的最大位移为\SI{0.355}{m},相差约5.6%。传统能源储库的最大位移比新能源储库大11.6%。因此,新能源储备库的热力学效应引起的变形较传统能源储备库更为显著,然而新能源储备库总体变形量比传统能源储备库的总体变形量小,一定程度说明高频注采未对盐岩储库的整体流变量造成较大影响,反而抑制了一部分变形量。

(3)在长期运行过程中,由于新能源储库的荷载循环周期短,温度传递时间有限,因此围岩中的高温(\SI{351.9}{K})小于气体的最高温(\SI{378}{K}),而传统能源储库围岩的最高温度与腔内气体最高温相等。

(4)长轴为\SI{70}{m},短轴为\SI{30}{m}的椭球体新能源储备库,运行10年后,长半轴温度影响范围为\SI{37}{m},短半轴温度影响范围为\SI{47}{m}。同等条件的传统能源储备库运行10年后,长半轴温度影响范围为\SI{47}{m},短半轴温度影响范围为\SI{58}{m}。相较于新能源储库,传统能源储库的温度影响范围更大,围岩最高温度更大,因此传统能源储库围岩更易因温度过高产生流变,更易出现多腔体之间温度场叠加。

(5)新能源储库考虑热力耦合的情况下腔顶和腔腰部分的最大Mises应力比未考虑热力耦合增大7%-10%,而腔腰部分的应力有所减小;传统能源储库考虑热力耦合的情况下各部分的最大Mises应力均比未考虑热力耦合小1%-2%。

(6)新能源储库各位置的最大Mises应力均大于传统能源储库对应位置的最大Mises应力,增长幅度可达18%-22%,因此循环周期对于储库运行的应力场影响十分明显。%\todoiZN{这里要系统分析,哪个部位大,几种工况下的增加幅度。CAES的压力大两倍取决于什么时候的取值,对比的应力是什么时候的应力。}

\section{展望}

(1)储库在正常工作状态下,气体不断压缩与释放的过程中温度时刻发生变化,与围岩一直处于热量交换状态。为了更精确的模拟该过程,应将热力学与固体力学在时间上的完全耦合,但理论研究、模拟软件和计算机算力等方面都难以达到要求,因此对耦合的过程进行了适当简化,忽略了工作情况下气体与围岩发生的热量交换。

(2)考虑到我国多为层状盐岩,计算模型中可以设置盐岩夹层,以探究含夹层盐岩地下储库长期运行过程中的热力耦合效应,模拟结果更贴近实际工程,更具参考价值。

(3)热力耦合下盐岩储气库的长期蠕变变形性质,对盐岩储气库稳定性的评价准则进行补充、完善。