\begin{thanks}
光阴荏苒,岁月如梭,三年的研究生生涯即将结束。回首往昔,感慨良多。在此向研究生期间对我的学习生活给予帮助的老师、同学和家人们致以最诚挚的感谢!

首先我要感谢的是范进教授,范老师学识渊博、治学严谨,为人谦和,待人真诚,本论文的完成离不开范老师的帮助。选题之初期范老师帮助我打开思路,明确研究方向;写作期间,范老师时常关心我的研究进展,耐心细致的纠正我的错误。大到文章结构逻辑,小到措辞文字,范老师的严谨细致总是让我十分敬佩。论文习作期间,不仅开阔了我的视野,丰富了我的思想,潜移默化中也影响了我行为处事的方式。

另外还要感谢张宁老师,从踏入南京理工大学开始,张老师在学习和工作上都给予了我很大的帮助。每当我在科研中遇到问题时,张老师都会耐心地对我进行指导,帮助我克服了一个又一个障碍。在这三年中,张老师还多次带领我们外出实践、参与项目试验,极大地丰富了我的眼界和工程实践经验,让我获益匪浅。
张老师精益求精的工作作风,严谨求实的治学态度,是我学习的榜样。学术之外,张老师为人谦和内敛,稳重细致,对我在科研之余的学习生活给予关心和帮助。在此对于二位老师在我论文选题、试验、撰写到最终定稿等方面给予的指导和帮助致以最崇高的敬意和诚挚的谢意。

感谢我的同门戴伟、罗小佳、王赐佳师兄、杨万里师弟以及我的三位室友对我学习和生活的帮助,是你们让我的研究生生涯变得更加丰富多彩。

此外,还要特别感谢我的家人们,是你们在背后的默默支持与付出,我才能全身心投入学习和科研中去,顺利完成这三年的研究生生涯

最后,感谢各位教授、专家在百忙之中对我论文的评阅,同时也感谢你们对我的论文提出宝贵的指导和修改意见。


特此致谢。
   

    \vskip 18pt

\end{thanks}